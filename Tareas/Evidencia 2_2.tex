\documentclass[10pt,a4paper]{article}
\usepackage[utf8]{inputenc}
\usepackage{amsmath}
\usepackage{amsfonts}
\usepackage{amssymb}
\usepackage{graphicx}
\usepackage[left=2cm,right=2cm,top=2cm,bottom=2cm]{geometry}
\author{Gabriel Alvarez}
\title{Arreglos y parametros de los Amplificadores clase A}
\begin{document}
\maketitle\section{Amplificadores clase A}
CLASIFICACIÓN:
1.Según el elemento activo:
Transistores unipolares: Emisor común (EC), Colector común (CC), Base común (BC) Transistores bipolares: FET o MOS(potencia) oTransistores especiales. 
Circuitos integrados: Operacionales y específicos de audio, video, instrumentación. 

2.Según el tipo de señal De CC: 
En fuentes de alimentación o para activación de actuadores(válvulas, motores, lámparas, relés..) 
De Señal: 
Baja frecuencia: 
amplificación de transductores para medida Media frecuencia 
Amplificación de voz o música(20Hz-20KHz) 
Alta frecuencia
Amplificación de video (15Hz-15MHz). 
Señal de radiofrecuencia>20KHz. 

3.Según la potenciaoDe pequeña señal: 
Etapas previas de amplificación o para corrientes débiles. 
De potencia: ültimas etapas de amplificación o para corrientes grandes. 
Clase A: No se recorta la señal. 
Clase B: La señal se recorta durante medio semiciclo. 
Clase C: La señal se recorta durante más de un semiciclo. 
Clase AB:La señal se recorta durante menos de un semiciclo 

4.Según las etapas de amplificaciónoMonoetapa: 
Simple, diferencial, realimentación. 
Multietapa 
Acoplamiento: Directo, RC, LC, con transformador.

\end{document}